\documentclass{article}

\usepackage{hyperref}
\usepackage{setspace}
\usepackage{fancyvrb}
\usepackage{tikz}
\usetikzlibrary{arrows,shapes,positioning}

\begin{document}

\DefineVerbatimEnvironment{pflcodeve}{Verbatim} {xleftmargin=3.0em,fontsize=\small}

\newenvironment{pflcode}
  {\VerbatimEnvironment \setstretch{0.8} \begin{pflcodeve}}
  {\end{pflcodeve} }

\newcommand{\true}             {\mathtt{t}}
\newcommand{\false}            {\mathtt{f}}
\newcommand{\pathsep}          { $\triangleright$ }
\newcommand{\tableline}        {\noalign{\hrule height 0.8pt}}
\newcommand{\optionsection}[1] {\subsection*{\texttt{#1}}}

\tikzstyle{nodestyle}   = [draw, thick, circle, minimum size=0.9cm]
\tikzstyle{bnedgestyle} = [-triangle 45,thick]

\setlength{\parskip}{\baselineskip}

\title{\Huge\textbf{LP$^2$ Manual}}

\author{Fabrizio Riguzzi$^1$ \\\texttt{fabrizio.riguzzi@unife.it} \and Riccardo Zese$^2$\\\texttt{riccardo.zese@unife.it}\\\\
$^1$ Dipartimento di Ingegneria -- Universit\`a di Ferrara\\
Via Saragat 1, 44122, Ferrara, Italy  \\
$^2$ Dipartimento di Matematica e Informatica -- Universit\`a di Ferrara\\
Via Saragat 1, 44122, Ferrara, Italy  \\
}


\date{\today}

\maketitle
\thispagestyle{empty}
\vspace{5cm}

\newpage



%------------------------------------------------------------------------------
%------------------------------------------------------------------------------
%------------------------------------------------------------------------------
%------------------------------------------------------------------------------
\section{Introduction}

LP$^2$ \cite{BelLamRig14-ICLP-IJ} performs inference via lifted variable elimination for probabilistic logic programming languages under the distribution semantics. 
LP$^2$ was developed by adapting the Generalized Counting First Order Variable Elimination (GC-FOVE) algorithm presented in \cite{gcfove} to this scope. 
LP$^2$ extends GC-FOVE by introducing two new operators, \textit{heterogeneous sum} and \textit{heterogeneous multiplication}.
LP$^2$ reads a file containing a ProbLog program and translates it to a Prolog Factor Language (PFL) file that is then processed by Yap.
PFL allows to describe first-order probabilistic factors. 

To perform inference with LP$^2$ you need to first translate ProbLog to PFL.
\section{Problog}
Problog is a probabilistic extension of Prolog that defines a probabilistic distribution over logic programs by specifying the probability of some facts.
A ProbLog program is composed by a set of rules (ordinary Prolog rules) and a set of probabilistic ground facts of the form
\begin{pflcode}
0.9::a(a1).
\end{pflcode}
Non ground facts can be represented as
\begin{pflcode}
0.7::c(X) :- x(X).

x(a1).
x(a2).
\end{pflcode}
where the \verb|x/1| predicate is used to define the domain of the variable \verb|X|. \verb|x/1|  should not depend from probabilistic facts. It can be defined
by facts and/or rules and represents a ``constraint''.

Next we show an example of a Problog program.
%esPoole
\begin{pflcode}
% clauses
e3:-e1,e2.

e1:-a(X),b(X),c(X),x(X).
e2:-a(X),b(X),c(X),x(X).

% probabilistic clauses
0.9::a(X) :- x(X).
0.8::b(X) :- x(X).
0.7::c(X) :- x(X).

% facts
x(a1).
x(a2).
\end{pflcode}

%%%%%
% :- CONSTRAINTS....
%%%%%

LP$^2$ needs to know which predicates are constraints in the ProbLog program so the following statement must be added at the beginning of the file
\begin{pflcode}
 :-contraints(<listOfContraints>).
\end{pflcode}
\texttt{<listOfContraints>} is a list of predicate specifications of the form \verb|p/a|.
Thus, the input file for the above program is
\begin{pflcode}
% contraints
:- constraints([x/1]).

% clauses
e3:-e1,e2.

e1:-a(X),b(X),c(X),x(X).
e2:-a(X),b(X),c(X),x(X).

% probabilistic clauses
0.9::a(X) :- x(X).
0.8::b(X) :- x(X).
0.7::c(X) :- x(X).

% facts
x(a1).
x(a2).
\end{pflcode}
\section{PFL Language}
A first-order probabilistic graphical model is described using parametric factors, commonly known as parfactors. The PFL syntax for a parfactor is

$$Type~~F~~;~~Phi~~;~~C.$$

Where,
\begin{itemize}
\item $Type$ refers the type of network over which the parfactor is defined. It can be \texttt{bayes} for homogeneous factor on directed networks, \texttt{markov} for homogeneous factor on undirected ones, or \texttt{het} for heterogeneous factors 
in which the convergent variables are assumed to be represented by the first atom in the factor’s list of atoms (added by LP$^2$).

\item $F$ is a comma-separated sequence of Prolog terms that will define sets of random variables under the constraint $C$. If $Type$ is \texttt{bayes}, the first term defines the node while the remaining terms define its parents.

\item $Phi$ is either a Prolog list of potential values or a Prolog goal that unifies with one. Notice that if $Type$ is \texttt{bayes} or \texttt{het}, this will correspond to the conditional probability table. Domain combinations are implicitly assumed in ascending order, with the first term being the 'most significant' (e.g. $\mathtt{x_0y_0}$, $\mathtt{x_0y_1}$, $\mathtt{x_0y_2}$, $\mathtt{x_1y_0}$, $\mathtt{x_1y_1}$, $\mathtt{x_1y_2}$).

\item $C$ is a (possibly empty) list of Prolog goals that will instantiate the logical variables that appear in $F$, that is, the successful substitutions for the goals in $C$ will be the valid values for the logical variables. This allows the constraint to be defined as any relation (set of tuples) over the logical variables.
\end{itemize}

Besides \texttt{het} parfactors, LP$^2$ adds also \texttt{deputy} parfactors that are needed for heterogeneous variable elimination and that have two atoms in 
$F$ and no $Phi$.
They represent identity factors between the ground instantiations of the atoms.


%%%%%
\section{Translator}
%%%%%
A ProbLog file is translated into the corresponding PFL file by the Yap library \verb|lp2| that can be loaded by
\begin{pflcode}
?- use_module(library(lp2)).
\end{pflcode}
Two different commands are available:
\begin{itemize}
 \item \texttt{translate/2} which takes as input the name of the ProbLog input file and the name of the PFL output file, both with the file extension.
For example\\
\texttt{?- translate('input.pb','output.pfl').}

\item \texttt{translate/1} that takes as input the name of the ProbLog input file with the file extension. The corresponding PFL file's name is the name of the input file with extension \verb|.pfl|.
For example\\
\texttt{?- translate('input.pb').}\\
\noindent The output file's name is 'input.pfl'.

\end{itemize}

The output of the translator for the program above is
\begin{pflcode}
% solver settings
:- use_module(library(pfl)).
:- set_solver(lve).

% flags settings
:- yap_flag(unknown,fail).
:- set_pfl_flag(use_logarithms,false).

% set verbosity level
%:- set_pfl_flag(verbosity,3).

bayes e3,e1,e2;[1.0,1.0,1.0,0.0,0.0,0.0,0.0,1.0];[].
deputy e1,e11p;[].
het e11p,a(X),b(X),c(X);[1.0,1.0,1.0,1.0,1.0,1.0,1.0,0.0,
                         0.0,0.0,0.0,0.0,0.0,0.0,0.0,1.0];[x(X)].
deputy e2,e21p;[].
het e21p,a(X),b(X),c(X);[1.0,1.0,1.0,1.0,1.0,1.0,1.0,0.0,
                         0.0,0.0,0.0,0.0,0.0,0.0,0.0,1.0];[x(X)].

bayes a(X);[0.1,0.9];[x(X)].
bayes b(X);[0.2,0.8];[x(X)].
bayes c(X);[0.3,0.7];[x(X)].
x(a1).
x(a2).
\end{pflcode}

\section{Querying}
Once the ProbLog file is translated into PFL, it can be loaded in Yap and queries can be issued as in PFL.
For example, suppose the above program is stored into \verb|test.pfl|, you can issue the command
\begin{pflcode}
yap -l test.pfl
\end{pflcode}
to load the program and 
\begin{pflcode}
?- e3(X).
\end{pflcode}
to compute the probability that \verb|e3| is true.
Some examples of ProbLog programs ready for translation is in folder \verb|packages/CLPBN/examples/lp2|.

\section{Papers}
E. Bellodi, E. Lamma, F. Riguzzi, V. Santos Costa, and R. Zese. \textit{Lifted variable elimination for probabilistic logic programming}. Theory and Practice
of Logic Programming, (special issue on ICLP 2014), 2014. PDF: \url{http://arxiv.org/abs/1405.3218}.


\begin{thebibliography}{1}

\bibitem{BelLamRig14-ICLP-IJ}
E.~Bellodi, E.~Lamma, F.~Riguzzi, V.~Santos~Costa, and R.~Zese.
\newblock Lifted variable elimination for probabilistic logic programming.
\newblock {\em Theory and Practice of Logic Programming}, (special issue on
  ICLP 2014), 2014.

\bibitem{gcfove}
Nima Taghipour, Daan Fierens, Jesse Davis, and Hendrik Blockeel.
\newblock Lifted variable elimination: Decoupling the operators from the
  constraint language.
\newblock {\em Journal of Artificial Intelligence Research}, 47:393--439, 2013.

\end{thebibliography}
\end{document}
